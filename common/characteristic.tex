
{\actuality} 
\ifsynopsis{Исследование поведения жидкости в магнитным полем является сложной задачей из-за высокой нелинейности уравнений, описывающих физические законы для этой задачи. Развитие численного моделирования позволяет проводить уникальные эксперименты по исследованию нестабильностей, которые невозможно воспроизвести в физическом эксперименте. Полученные результаты позволят оценить количественно и качественно воздействия ряда магнитогидродинамических эффектов на поведение потока жидкости. Эти знания дополнят классическую теорию МГД, разработанную с рядом допущений.
}
\else
Этот абзац появляется только в~диссертации.
Через проверку условия \verb!\!\verb!ifsynopsis!, задаваемого в~основном файле
документа (\verb!dissertation.tex! для диссертации), можно сделать новую
команду, обеспечивающую появление цитаты в~диссертации, но~не~в~автореферате.
\fi
% {\progress}
% Этот раздел должен быть отдельным структурным элементом по
% ГОСТ, но он, как правило, включается в описание актуальности
% темы. Нужен он отдельным структурынм элемементом или нет ---
% смотрите другие диссертации вашего совета, скорее всего не нужен.

{\aim} данной работы является изучение механизмов возникновения, подавления и протекания турбулетных потоков проводящей немагнитной жидкости под воздействием бегущего магнитного поля в металлургических магнитогидродинамических насосах с учетом специфики их эксплуатации, а также исследование влияния не устойчивых режимов на основные характеристики магнитогидродинамических насосов.

Для~достижения поставленной цели необходимо было решить следующие {\tasks}:
\begin{enumerate}[beginpenalty=10000] % https://tex.stackexchange.com/a/476052/104425
  \item Разработать алгоритм расчета связанных задач магнитного, гидродинамического и температурного полей для потоков жидкости в каналах.
  \item Провести верификацию данного алгоритма с помощью тестовых задач и экспериментальных данных.
  \item Исследовать характер течений проводящей жидкости под воздействием бегущего магнитного поля с помощью идеализированных и упрощенных численных моделей. 
  \item Оценить влияние неравномерного профиля скоростей на распределение  электромагнитных усилий. 
  \item Исследовать влияние наличия неустойчивых потоков на основные характеристики индукционных насосов.
  \item Рассмотреть влияние термогравитационной конвекции на состояние потоков.  
\end{enumerate}


{\novelty}
\begin{enumerate}[beginpenalty=10000] % https://tex.stackexchange.com/a/476052/104425
  
  \item Разработан программный код для реализации алгоритмов по расчету гидродинамических процессов  несжимаемой неньютоновской жидкости под воздействием магнитного поля с использованием открытых пакетов OpenFOAM, Elmer и библиотеку связки между ними EOF-library.
    
  \item Также разработан алгоритм расчета потоков проводящей жидкости мультидисциплинарных задач естественной конвекции для сжимаемых и несжимаемых жидкостей под воздействием гармонического магнитного поля с возможностью учитывать фазовый переход одно- и двухкомпонетных сплавов. 
  
  \item Разработанные програмные прцоедуры по реализации численных расчетов магнитогидродинамических прцоессов были верифицированы на экспериментальной установке перемешивателя металла с помощью бегущего магнитного поля и численными тестовыми задачами, такими как вариации задачи Гартмана. 
  
  \item Был разработан код-обертка для автоматиации настройки моделей с возможностью проведения параметрических исследований, автоматической обработки результатов, проведения предварительных расчетов для начальных условий и рядом дополнительных опций, позволяющих существенно экономить время настройки модели. 
    
  \item Было выполнено оригинальное исследование возникновение неустойчивых состояний потока жидкости под воздействием бегущего магнитного поля 
  в диапазоне чисел Гартмана от 0 до 8000, Стюарта от 0 до 100, Рейнольдса от 100 до 55 000, магнитного Рейнольдса от очень малых значений до десятков. В данном исследовании сравниваются результаты распределения потока скоростей под идеализированным и неидеализированным магинтными полями. Проведен анализ влияния неустойчивых режимов на расходо-напорную характеристику линейного индукционного насоса. 
  
  \item Проведена оценка влияния потоков естественной конвекции на основные потоки вызванные разницей давлений на входе и выходе и приложенным внешним магнитным полем в безразмерном прямоугольном канале находящемся. Это исследование выполнено для безразмерных чисел \fixme{Гартмана, Стюарта, гидродинамического и магнитного Рейнольдса.} 
  
   
\end{enumerate}

{\influence} Процесс и подходы к расчетам и проектированию МГД-устройств могут быть изменены и доработан на основе полученных в этой работе результатов. Также понимание возникновения, протекания и подавления явления нестабильности возможно и не позволит решить проблему создания прототипов индукционных насосов, работающих выше 8 $бар$ при расходе жидкости от 3 \si{\metre^3} \fixme{$м^{3}/с$}, но предоставит возможность посмотреть на нее под другим углом и открыть новые подходы к ее решению. Разработанные рекомендации в процессе проекта по настройке моделей задач магнитной гидродинамики позволят снизить требования к вычислительной технике. Проведенный сравнительный анализ влияния краевых эффектов на поведение потока позволяет понять какие особенности  необходимо учитывать в математических моделях. Один из ключевых факторов к пониманию снижения эффективности МГД насосов это осознание возникновения и умение предсказывать неустойчивые состояния гидродинамической системы, которому уделяется большая часть этой работы. Впервые, в работе представлены карты состояний потока жидкости от чисел подобия для двухмерных и трехмерны случаев, доработка и расширение этих карт состояний в будущем позволит значительно облегчить и ускорить процесс проектирования МГД-устройств.  



{\methods} Рассматриваемая проблема является мультидисциплинарной задачей и требует совместного анализа магнитного поля, гидродинамики и, в ряде задач, термических явлений. Расчет магнитных полей  проводился с помощью метода конечных элементов, реализованного в  программе с окрытым кодом Elmer и использлванием гармнической <<$A-\varphi$>> формулировки. Расчет полей давления и скорости производился с  помощью метода конечных объемов, который в большей степени подходит для численного решения уравнений Навье-Стокса, используя специализированную программу с открытой лицензией для гидродинамических расчетов OpenFOAM. Температурные поля и фазовый состояния вещества, связанные с потоками жидкости, рассчитывались в OpenFOAM, а расчеты температурных режимов в магнитной системе(катушки, магнитопровод и т.д.) производились в Elmer. В ходе анализа ряда явлений учитывались особенности влияние гидродиномического процессов на магнитные и наоборот. Для осуществления данной опции бал реализован обмен данными с помощью message pass interface технологии, используя библиотеку EOF-library. Программы Elmer и OpenFOAM являются бесплатными продуктами с открытым кодом, но у них отсутствуют графические интерфейсы, а также вcтроенные инструменты преднастройки и постобработки результатов моделей, поэтому для решения данной проблемы была написана библиотека PyFoamRun для настройки моделей, запуске их на расчет, проведения параметрических исследований, автоматизированного создания сеток модели, обработки результатов и ряда других опций. Верификация разработанного кода была проведена с помощью сравнения подобных моделей, реализованных в программах Comsol Multiphysics на основе метода конечных элементов, и Ansys на основе метода конечных объемов с помощью интерфейса магнитнаягидродинамика. Также эти модели были верифицированы на экспериментальной установке МГД-перемешивателя жидкого галлия с бегущим магнитным полем и путем сравнения поля скоростей, измеряемого доплеровским датчиком. 


{\defpositions}
\begin{enumerate}[beginpenalty=10000] % https://tex.stackexchange.com/a/476052/104425
  \item Верифицированные численные модели  для расчета связанных задач гидродинамики, магнитного и температурного полей и возможностью учета фазвого перехода, разработанные на основе метода коненых эементов и объемов. 
  \item Алгоритмы для расчета связанных задач в открытых пакетах с возможностью автоматической настройки модели с помощью дополнительного кода--обертки. 
  \item Полученные закономерности влияния магнитных краевых эффектов на устойчивость потока жидкости и возникающих из-за них гидродинамических явлений.
  \item Соотношения, описывающие устойчивость потока жидкости в прямоугольных каналах в зависимости от чисел подобия. 
  \item Результаты влияния неустойчивых режимов работы магнитогидродинамических насосов на основные их характеристики. 
  \item Количественная оценка влияния тепловых явлений на поведение потока жидкости в в безразмерной постановки задачи для прямоугольном канала. 
  
\end{enumerate}


{\probation}
Основные результаты работы докладывались~на:
\begin{enumerate}[beginpenalty=10000] % https://tex.stackexchange.com/a/476052/104425
    \item IEEE NW Russian Young Researches in Electrical and Electronics and Electronic Engineering Conference, СПбГЭТЕУ <<ЛЭТИ>>, Санкт-Путербург, 2017, 2018, 2019, 2020. 
    \item XXII Зимняя школа по механике сплошных сред, ИМСС УРО РАН, г. Пермь, 2021
    \item XIX International UIE Congress on Evolution and New Trends in Electrothermal Processes, Плзень, Чехия, 2021. 
    \item X International Conference Electromagnetic processing of materials, Рига, Латвия, 2021.
    \item Четвертая Российская конференция по магнитной гидродинамике (РИМГД-21), ИМСС УРО РАН, Пермь, 2021.
    \item International Symposium on Heating by Electromagnetic Source (HSE-19), Padua, Italy.
    \item XXI Международная научная конференция <<Проблемы управления и моделирования в сложных системах>>, Самара, 2019.
    \item Международная конференция <<Актуальные проблемы электромеханики и электротехнологий>>, Екатеринбург, 2017, 2020. 
    \item Всеросийская конференция <<Наука. Технология. Инновации>>, Новосибирск, 2018. 
    \item VI Международный сименар  <<European Seminar on Computing>>, Пльзень, Чехия, 2018.
    \item Международная конференеция <<Computational Problems of Electrical Engineerin>>, Чехия, Кутна-Гора, 2017. 
\end{enumerate}

\ifnumequal{\value{bibliosel}}{0}
{%%% Встроенная реализация с загрузкой файла через движок bibtex8. (При желании, внутри можно использовать обычные ссылки, наподобие `\cite{vakbib1,vakbib2}`).
    {\publications} Основные результаты по теме диссертации изложены
    в~XX~печатных изданиях,
    X из которых изданы в журналах, рекомендованных ВАК,
    X "--- в тезисах докладов.
}%
{%%% Реализация пакетом biblatex через движок biber
    \begin{refsection}[bl-author, bl-registered]
        % Это refsection=1.
        % Процитированные здесь работы:
        %  * подсчитываются, для автоматического составления фразы "Основные результаты ..."
        %  * попадают в авторскую библиографию, при usefootcite==0 и стиле `\insertbiblioauthor` или `\insertbiblioauthorgrouped`
        %  * нумеруются там в зависимости от порядка команд `\printbibliography` в этом разделе.
        %  * при использовании `\insertbiblioauthorgrouped`, порядок команд `\printbibliography` в нём должен быть тем же (см. biblio/biblatex.tex)
        %
        % Невидимый библиографический список для подсчёта количества публикаций:
        \printbibliography[heading=nobibheading, section=1, env=countauthorvak,          keyword=biblioauthorvak]%
        \printbibliography[heading=nobibheading, section=1, env=countauthorwos,          keyword=biblioauthorwos]%
        \printbibliography[heading=nobibheading, section=1, env=countauthorscopus,       keyword=biblioauthorscopus]%
        \printbibliography[heading=nobibheading, section=1, env=countauthorconf,         keyword=biblioauthorconf]%
        \printbibliography[heading=nobibheading, section=1, env=countauthorother,        keyword=biblioauthorother]%
        \printbibliography[heading=nobibheading, section=1, env=countregistered,         keyword=biblioregistered]%
        \printbibliography[heading=nobibheading, section=1, env=countauthorpatent,       keyword=biblioauthorpatent]%
        \printbibliography[heading=nobibheading, section=1, env=countauthorprogram,      keyword=biblioauthorprogram]%
        \printbibliography[heading=nobibheading, section=1, env=countauthor,             keyword=biblioauthor]%
        \printbibliography[heading=nobibheading, section=1, env=countauthorvakscopuswos, filter=vakscopuswos]%
        \printbibliography[heading=nobibheading, section=1, env=countauthorscopuswos,    filter=scopuswos]%
        %
        \nocite{*}%
        %
        {\publications} Основные результаты по теме диссертации изложены в~\arabic{citeauthor}~печатных изданиях,
        \arabic{citeauthorvak} из которых изданы в журналах, рекомендованных ВАК\sloppy%
        \ifnum \value{citeauthorscopuswos}>0%
            , \arabic{citeauthorscopuswos} "--- в~периодических научных журналах, индексируемых Web of~Science и Scopus\sloppy%
        \fi%
        \ifnum \value{citeauthorconf}>0%
            , \arabic{citeauthorconf} "--- в~тезисах докладов.
        \else%
            .
        
        \fi%
        % К публикациям, в которых излагаются основные научные результаты диссертации на соискание учёной
        % степени, в рецензируемых изданиях приравниваются патенты на изобретения, патенты (свидетельства) на
        % полезную модель, патенты на промышленный образец, патенты на селекционные достижения, свидетельства
        % на программу для электронных вычислительных машин, базу данных, топологию интегральных микросхем,
        % зарегистрированные в установленном порядке.(в ред. Постановления Правительства РФ от 21.04.2016 N 335)
    \end{refsection}%
    \begin{refsection}[bl-author, bl-registered]
        % Это refsection=2.
        % Процитированные здесь работы:
        %  * попадают в авторскую библиографию, при usefootcite==0 и стиле `\insertbiblioauthorimportant`.
        %  * ни на что не влияют в противном случае
        
    \end{refsection}%
        %
        % Всё, что вне этих двух refsection, это refsection=0,
        %  * для диссертации - это нормальные ссылки, попадающие в обычную библиографию
        %  * для автореферата:
        %     * при usefootcite==0, ссылка корректно сработает только для источника из `external.bib`. Для своих работ --- напечатает "[0]" (и даже Warning не вылезет).
        %     * при usefootcite==1, ссылка сработает нормально. В авторской библиографии будут только процитированные в refsection=0 работы.
}



    \nocite{Smolyanov:CAMWA:2019}
    \nocite{Smolyanov:NMSS:2021}
    
    \nocite{Smolianov:MHD1:2021}
    \nocite{Smolianov:MHD2:2021}
    
    \nocite{Shvydkiy:MHD:2017}
    \nocite{Sarapulov:MHD:2017}
    \nocite{Bolotin:MHD:2017}
    
    \nocite{Sarapulov:ElTech:2018}
    \nocite{Smolyanov:QElTech:2018}
    \nocite{Sarapulov:ElMech:2019}
    
    \nocite{Sarapulov:ACTA:2018}
    \nocite{Bolotin:ACTA:2018}
    
    \nocite{Shmakov:ElConRus:2020}
    \nocite{Smolyanov:ElConRus:2020}
    \nocite{Tarchutkin:ElConRus:2020}
    
    \nocite{Smolyanov:ELEKTRO:2018}
   
    \nocite{Smolyanov:ElconRus:2018}
    \nocite{Smolyanov:ElConRus:2018_lev}
    \nocite{Bolotin:ElconRus:2017}
    \nocite{Smolyanov:ElconRus:2017}
    
    
    \nocite{Shvidkiy:ElconRus:2016}


